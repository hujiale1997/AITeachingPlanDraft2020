\begin{pre}
	\thispagestyle{empty}
\begin{center}
  {\kaishu{材料来自于网上,如有侵权,请联系ocnzhao@163.com予以纠正,THX}}
\end{center}
\begin{center}
%		{\kaishu{人在春风和气中}}
\begin{figure}[htbp]
	\centering
	\includegraphics[width=0.5\textwidth]{Imagetreemath.png}
\end{figure}
\end{center}
\begin{center}
    本教案二维码下载地址: \qrcode[height=1in]{https://github.com/zggl/AITeachingPlanDraft2020/blob/master/AIMaster2020.pdf}
\end{center}
\begin{center}
%		{\kaishu{人在春风和气中}}
\begin{figure}[htbp]
	\centering
	\includegraphics[width=0.46\textwidth]{ImageAIwwl01.png}
    \includegraphics[width=0.45\textwidth]{ImageAIGQ.png}\\
    \includegraphics[width=0.45\textwidth]{ImageAILYR.png}
    \includegraphics[width=0.45\textwidth]{ImageAILDY.png}\\
\end{figure}
\begin{figure}[htbp]
    \includegraphics[width=0.45\textwidth]{ImageAISZZ.png}
    \includegraphics[width=0.45\textwidth]{ImageAIJQXX.png}
\end{figure}

史忠植——2013年凭借“拓展知识工程核心理论、创新分布智能理论基础、构建智能科学理论体系”成果, 荣获第三届吴文俊人工智能科学技术奖成就奖.

周志华——教育部“长江学者”特聘教授,国家杰出青年基金获得者;南京大学人工智能学院院长. 有一个旧称叫国立东南大学.
\end{center}

\href{https://www.sciencedaily.com/}{Science daily}

\paragraph{在线课程} 课程相关的电子材料存放平台和链接地址。

\href{https://gitee.com/zggl/AITeachingPlanDraft2020}{码云平台上的课程电子版和代码}

\href{https://github.com/zggl/AITeachingPlanDraft2020}{Github平台上的课程电子版和代码}

\href{https://ke.qq.com/webcourse/index.html?cid=1086628&term_id=101182654&lite=1&from=800021724#taid=5467185&vid=5285890799477964181}{1——人工智能基础-第一次 基础简介}

\href{https://ke.qq.com/webcourse/index.html?cid=1086628&term_id=101182654&lite=1&from=800021724#taid=8423279&vid=5285890799803709554}{2——人工智能基础-第二次-发展和应用}

\href{https://github.com/zggl/AITeachingPlanDraft2020/blob/master/3-\%E4\%BA\%BA\%E5\%B7\%A5\%E6\%99\%BA\%E8\%83\%BD\%E7\%AC\%AC\%E4\%BA\%8C\%E7\%AB\%A0\%E7\%AC\%AC\%E4\%B8\%80\%E6\%AC\%A1\%E5\%B9\%BB\%E7\%81\%AF\%20\%E4\%BA\%BA\%E5\%B7\%A5\%E6\%99\%BA\%E8\%83\%BD\%E7\%9A\%84\%E7\%9F\%A5\%E8\%AF\%86\%E8\%A1\%A8\%E7\%A4\%BA.pdf}{3——人工智能——知识表示}

\href{https://github.com/zggl/AITeachingPlanDraft2020/blob/master/4-\%E4\%BA\%BA\%E5\%B7\%A5\%E6\%99\%BA\%E8\%83\%BD\%E7\%AC\%AC\%E4\%B8\%89\%E7\%AB\%A0\%E7\%AC\%AC\%E4\%B8\%80\%E6\%AC\%A1\%20\%E4\%BA\%BA\%E5\%B7\%A5\%E6\%99\%BA\%E8\%83\%BD\%E7\%9A\%84\%E7\%9F\%A5\%E8\%AF\%86\%E6\%8E\%A8\%E7\%90\%86\%E6\%96\%B9\%E6\%B3\%95.pdf}{4——人工智能第三章——知识推理方法}
\href{https://edu.tipdm.org/notification?id=32302}{泰迪云课堂}

\href{http://speech.ee.ntu.edu.tw/~tlkagk/courses_ML20.html}{台大李宏毅老师的机器学习课程-2020}

\href{https://edu.tipdm.org/classroom/122/courses}{泰迪-深度学习原理及编程实现\_人邮版}

\href{https://deeplearn.org/}{Deep Learning Monitor}
%\vspace*{5\baselineskip}
%\centerline{\includegraphics[scale=0.6]{example/gzh.jpg}}
%\centerline{\fontsize{26pt}{26pt} 微信公众号}

\href{http://alexlenail.me/NN-SVG/}{网络绘图工具} 

\href{}{tensorwatch, python的扩展包,作者随机产生的loss和accuracy,经过这个package的调用,将其动态显示成为图表}

\href{ConvNetDraw(卷积神经网络),配置命令的CNN神经网络画图工具,开发者是香港的一位程序员。}{https://cbovar.github.io/ConvNetDraw/}

48学时可以讲授到第七章.

\end{pre} 